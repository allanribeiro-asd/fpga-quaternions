\documentclass [12pt,oneside] {article}
\include{packages} % incluir pacotes e configurações
\begin {document}

%==========================================================================

\title {Calculo do deslocamento de corpos usando Quatérnions}

\author{
Allan Ribeiro - 2037807\\
Lucas P. S. Leyser - 1860275\\
}

\date {21 de Dezembro de 2022}

\maketitle

%==========================================================================


\section{Introdução}

%==========================================================================

	A intenção era fazer o uso de FPGA para cálculo do
deslocamento de uma partícula no tempo fazendo uso da abstração de
quaternios, que evita problemas algébricos, diferentemente da
abordagem dos ângulos de Euler.

	Fez se o uso dos dados de giroscópio e acelerômtetros de um
smartphone, e suas respectivas leituras descritas em um arquivo csv, 
fazendo-se a amostragem em tempo da leitura da aceleração das variações, 
em eixo x, y e z. Os dados são então filtrados de string para ponto flutuante, 
processados por um socket, escrito em python, que por sua vez realiza 
a conexão com a placa.

	Na placa Cyclone II, foram escritos arquivos em VHDL que fazem
a abstração do uso de registradores para guardar um quaternion, sendo
descrito como o valor normalizado de dimensão única. Foram usados 3
registradores, um de controle das operações, um como buffer, e outro que acumula o resultado das operações, devolvendo o resultante da multiplicação de todos os quaternios recebidos.

	Sem sucesso, o presente relatório dá visão para problemas que
futuros desenvolvedores terão, se usada a mesma abordagem. Dentre os
inúmeros, lista-se o empecilho principal: uma operação efetiva de
multiplicação. Os procedimentos são detalhados abaixo.

	Alguns pontos importantes que leitores devem saber é que a
estrutura para a multiplicação dos quaternios em sí está pronta,
sendo necessária a correção das operações. Um testbench também foi
elaborado para uso com os sources disponíveis, e está funcional. O
parsing dos dados também foi realizado, e as respectivas funções estão
prontas, além de um envio efetivo realizado.

\section{Desenvolvimento do projeto}

\subsection{Aquisição de dados}

	Para amostrar os dados a serem usados no projeto, foi feito
uso do phyphox[1], e sua funcionalidade de leitura de diversos
sensores de um dispositivo que rode android. Este aplicativo permite a
visualização em tempo real das métricas de diversos sensores, além da
captura e exportação dos dados em CSV, um formato padrão em texto de
armazenamento de dados.

	Os dados foram exportados e disponibilizados num arquivo, que
por sua vez é processado pelo servidor, que mais tarde alimenta a
placa com os mesmos. A formatação transforma a 4-tupla de valores de
string (tempo, qx, qy e qz) em pontos fixos, com a mesma
quantidade de casas decimais. Isso é importante para facilitar a
entrada e armazenamento dos dados na placa. Mais tarde, serão
codificados novamente em string para envio efetivo pela rede.

\subsection{Conexão de rede}

	A conexão em sí foi construída em 2 partes. Primeiro, um
servidor em python espera por uma conexão em socket, que roda em um
computador à parte, aqui referido como servidor. O servidor, após filtrar
os pontos, os envia em 4-tupla à placa, um após o outro. Uma vez
enviados, espera por uma tupla do mesmo tipo vinda da placa.

	A conexão TCP/IP com a placa Altera DE2 Cyclone II, cliente,
foi projetada com base no projeto "VHDL FPU" realizado pelo aluno
Luciano Bonzatto , no segundo semestre de 2022. Estabelecida com
sucesso, é necessário a especificação do endereço de servidor a
contatar.

	Importante notar que a placa recebe uma string, e não uma
tupla. Futuros esforços devem ser concentrados na conversão da string
recebida, novamente para uma tupla, e em seguida em ponto fixo.

	Os dados são, por fim, escritos em 3 registradores internos da
placa, um como indicador de operação (load e mult), um como buffer, e
um terceiro acumulador de operações, que é lido ao final das operações
com todos os quaternios desejados, esse é o resultado devolvido ao servidor.

\subsection{Multiplicação}

	Para o cálculo da multiplicação dos quaternos, foram
utilizados sinais do tipo "sfixed" que utiliza um ponto fixo
determinando qual a parte inteira e qual a parte decimal do valor
recebido, que é então armazenado em um registrador de 32 bits. Isso
resulta em uma precisão de $2^{-15}$ nas operações.

	A operação, infelizmente, apresenta um problema: em teste de
mesa, a tentativa de multiplicar 0.5 por 0.5 resulta no deslocamento
errôneo dos bits.

\begin{footnotesize} \begin{verbatim}

	0.5 * 0.5 = 0b0010 * 0b0010 = 0b0001 = 0.25 (caso ideal)

	0.5 * 0.5 = 0b0010 * 0b0010 = 0b000001 = 0.0625 (caso real)

\end{verbatim} \end{footnotesize}

Mas o caso é isolado. Para outros casos, o resultado é caótico à
primeira vista. Na tentativa de resolver o problema, foram feitos
testes de outras operações, e os resultados também são incorretos,
porém consistentes.

	Devido à falta de suporte a esse tipo de sinal por parte do
Qsys, foi preciso utilizar portas do tipo \texttt{std\_logic\_vector}
para o recebimento dos valores e a própria FPGA faz a cópia desse
valor para um sinal do tipo \texttt{sfixed}. Partimos do pressuposto que,
como os valores são sinalizados e entre 0 e 1, sempre teríamos dois
bits iniciais, o primeiro para o sinal e o segundo para a parte
inteira do valor. O restante é a parte decimal do valor, ou seja,
utilizando um sinal com o ponto fixo entre o segundo e terceiro MSB do
\texttt{std\_logic\_vector} a cópia pode ser feita diretamente, bit a
bit.

	Foi feita a tentativa de abstrair outra forma de cálculo:
receber os dados em binário, e fazer a operação bit a bit. Outro
problema apareceu aqui, advindo da abstração usada para fazer a multiplicação entre os pontos. No caso de
um dos números ser negativo, a multiplicação em complemento de 2
não foi implementada com sucesso. Resolvido este problema, a
multiplicação deve funcionar efetivamente.

	Outra tentativa consistiu em trabalhar com inteiros,
fazendo-se shift dos bits, multiplicando-os, e retornando à posição
original. Ocorreram os mesmos problemas da abordagem por complemento
de 2.


%==========================================================================
\section{Conclusão e recomendações gerais}
%==========================================================================

	O projeto teve avanços, e está perto de estar completo.
Novamente, a conexão de rede funciona. A multiplicação também, mas existem erros na
operação.

	Como recomendações gerais para futuros uso do projeto, troque
a versão da placa para evitar problemas de licença, problemas de
compilação e geração de componentes/circuitos (Qsys).

	A forma como a multiplicação foi feita é dependente da
biblioteca ieee\_proposed, representada pelos arquivos
\texttt{fixed\_float\_types\_c.vhdl, float\_pkg\_c.vhdl} e
\texttt{fixed\_pkg\_c.vhdl}. Os três tem problemas com versões de
vhdl, cujo dialeto deve ser 2008.

	As bibliotecas devem ser inserida como arquivos do projeto no
Quartus (caso simulação em RTL), no Qsys (para geração dos
componentes), e compiladas internamente do ModelSim (para simulação em
RTL).

	A multiplicação em binário na FPGA é considerada, por muitas
referências, como uma área obscura para se implementar. Sem sucesso,
algumas recomendações são: implementar uma lógica de complemento de 2
funcional, uso de outro tipo preferível (real não é sintetizável), ou
mesmo uso de outros formátos numéricos mais simples de representar.


[1] - Página Principal PhyPhox https://phyphox.org/

\end{document}
%==========================================================================

